\documentclass[a4paper,12pt]{ltjsarticle}

\usepackage{luatexja}

\usepackage{graphicx}

\usepackage[svgnames]{xcolor}

\usepackage{here}

\usepackage{floatflt}

\usepackage{mathtools}

\usepackage{mathrsfs}

\usepackage{physics}

\usepackage{braket}

\usepackage{enumitem}

\usepackage{url}

\usepackage{amsmath, amssymb, amsthm, bm}

\usepackage[unicode, hidelinks, pdfusetitle]{hyperref}

\usepackage{nameref}

\usepackage{cleveref}

\usepackage{tcolorbox}

\usepackage{comment}

\usepackage{framed,color}

\usepackage{anyfontsize}

\usepackage[top=30truemm,bottom=30truemm,left=25truemm,right=25truemm]{geometry}

\hypersetup{
	colorlinks=true,
	citecolor=blue,
	linkcolor=teal,
	urlcolor=orange,
}

\newtheoremstyle{break}
  {\topsep}{\topsep}%
  {}{}%
  {\bfseries}{}%
  {\newline}{}%
\theoremstyle{break}

\newtheorem{thm}{Theorem}[section]

\newtheorem{defn}[thm]{定義}
\newtheorem{eg}[thm]{例}
\newtheorem{thrm}[thm]{定理}
\newtheorem{ax}[thm]{公理}
\newtheorem{lem}[thm]{補題}
\newtheorem{cor}[thm]{系}
\newtheorem{fact}[thm]{事実}
\newtheorem{rem}[thm]{注意}
\newtheorem{req}[thm]{要請}
\newtheorem{res}[thm]{結果}
\newtheorem*{prf}{証明}

\makeatletter
\newenvironment{pr}[1][\proofnam]{\par
%\newenvironment{Proof}[1][\Proofname]{\par
  \normalfont
  \topsep6\p@\@plus6\p@ \trivlist
  \item[\hskip\labelsep{\itshape #1}\@addpunct{\bfseries}]\ignorespaces
}{%
  \endtrivlist
}
\newcommand{\proofnam}{\underline{Derivation.}}
\makeatother

% 圏論的量子力学
\newcommand{\cset}{\mathbf{Set}}
\newcommand{\fset}{\mathbf{FSet}}
\newcommand{\rel}{\mathbf{Rel}}
\newcommand{\frel}{\mathbf{FRel}}
\newcommand{\vect}{\mathbf{Vect}}
\newcommand{\fvect}{\mathbf{FVect}}
\newcommand{\hilb}{\mathbf{Hilb}}
\newcommand{\fhilb}{\mathbf{FHilb}}
\newcommand{\mon}{\mathbf{Mon}}
\newcommand{\cmon}{\mathbf{CMon}}
\newcommand{\pre}{\mathbf{Pre}}
\newcommand{\mbc}{\mathbb{C}}
\newcommand{\mbr}{\mathbb{R}}
\newcommand{\mrr}{\mathrm{R}}
\newcommand{\prob}{\mathrm{Prob}}
\newcommand{\Ob}{\mathrm{Ob}}
\newcommand{\Mor}{\mathrm{Mor}}
\newcommand{\Hom}{\mathrm{Hom}}
\newcommand{\Op}{\mathrm{op}}
\newcommand{\xr}[1]{\xrightarrow{#1}}
\newcommand{\xrd}[1]{\xrightarrow{#1}}
\newcommand{\rr}[2]{\xrightrightarrows{#1}{#2}}
\newcommand{\dom}{\mathrm{dom}}
\newcommand{\cod}{\mathrm{cod}}
\newcommand{\id}{\mathrm{id}}
\newcommand{\eva}{\mathrm{ev}}
\newcommand{\coev}{\mathrm{coev}}
\newcommand{\osi}[1]{\overset{#1}{\sim}}
\newcommand{\iso}[1]{\underset{#1}{\sim}}
\newcommand{\ci}{\circ}
\newcommand{\al}{\alpha}
\newcommand{\la}{\lambda}
\newcommand{\si}{\sigma}
\newcommand{\vp}{\varphi}
\newcommand{\vP}{\varPhi} 
\newcommand{\ve}{\varepsilon}
\newcommand{\da}{\dagger}
\newcommand{\dav}{\dashv} 
\newcommand{\ra}{\Rightarrow}
\newcommand{\lar}{\Leftarrow}
\newcommand{\mt}{\mapsto}
\newcommand{\ot}{\otimes}
\newcommand{\bigot}{\bigotimes}
\newcommand{\ti}{\times}
\newcommand{\opl}{\oplus}
\newcommand{\bigop}{\bigoplus}
\newcommand{\am}{\amalg}
\newcommand{\tra}{\mathrm{Tr}}
\newcommand{\dime}{\mathrm{dim}}
\newcommand{\scr}{\scriptstyle}

% 解析力学
\newcommand{\dq}{\dot{q}}
\newcommand{\dQ}{\dot{Q}}
\newcommand{\el}[1]{\mathscr{E}_i[{#1}]}
\newcommand{\elk}[2]{\mathscr{E}_{#1}[{#2}]}
% \newcommand{\el*}{\mathscr{E}_i[L^*]}
\newcommand{\cl}{c_L}
\newcommand{\dcl}{\dot{c}_L}
\newcommand{\tcl}{\tilde{c}_L}
\newcommand{\del}[1]{\frac{\partial}{\partial{#1}}}
\newcommand{\Del}[1]{\frac{d}{d{#1}}}
\newcommand{\ddel}[2]{\frac{\partial{#1}}{\partial{#2}}}
% \newcommand{\d2ddel}[3]{\frac{\partial^2{#1}}{\partial{#2}\partial{#3}}}
\newcommand{\ddddel}[2]{\frac{\partial^2{#1}}{\partial{#2}^2}}
\newcommand{\DDel}[2]{\frac{d{#1}}{d{#2}}}
\newcommand{\deldq}{\left( \frac{\partial L}{\partial \dot{q}^i} \right)}
\newcommand{\tq}{\tilde{q}}
\newcommand{\dtq}{\dot{\tilde{q}}}

% 接触幾何
\newcommand{\const}{\mathrm{const}}

\makeatletter
\@addtoreset{equation}{section}
\makeatother
\numberwithin{equation}{section}

\definecolor{lightgray}{rgb}{0.75,0.75,0.75}

\renewenvironment{leftbar}{%
	\def\FrameCommand{\textcolor{lightgray}{\vrule width 3pt} \hspace{3pt}}% 
	\MakeFramed {\advance\hsize-\width \FrameRestore}}%
{\endMakeFramed}

\renewcommand{\appendixname}{Appendix}

\title{微分形式による熱力学}
\author{よの}
\date{\today}

\begin{document}

\maketitle

\newpage

\section*{まえがき}

このpdfはHermann `Geometry, Physics, and Systems'のChapter6をまとめたものである. 
本の説明が正しくない箇所はこのpdf内で訂正を入れている. 
該当箇所の注釈で詳しい説明をする. 

\newpage

\tableofcontents

\newpage

\section{滑らかな単一熱平衡系}

熱力学を微分形式の言葉を用いて定義しなおす. 
これが従来の熱力学を再現することを確かめる. 

\subsection{Gibbs 1-formと滑らかな単一熱平衡系}

\begin{defn}[熱系]
  多様体$M$を5次元Euclid空間$\mbr^5$とする. 
    \footnote{
      今は大域的に$\mbr^5$がとれると仮定する. 
      局所的な議論には接触幾何が必要である. 
    }
  %Mを5次元多様体として, 局所的にR^5とするのはcontact geometryの章でする. 
  $M$のCartesian座標を$(U,T,S,P,V)$とする. 
  この$M$を熱系という. \\
  $U$を内部エネルギー(internal energy), $T$を温度(temperature), $S$をエントロピー(entropy), $P$を圧力(pressure), $V$を体積(volume)という. 
  \footnote{
    モル質量$n$と化学エネルギー$N$(chemical energy)等を付け加えてもよい. 
    重要なのはこの多様体が奇数次元であることである.
    }
\end{defn}

熱力学第一法則が微分形式を用いて
\begin{align*}
  dU - TdS + PdV 
  = 0
\end{align*}
で表されることから左辺の1-formを改めて定義して, この微分形式のpulbackで熱系を特徴付けられないか考えるのは自然である. 

\begin{defn}[Gibbs 1-form]
  $M$上の1-form $\theta$を
  \begin{align}
    \label{gibbs}
    \theta 
    := dU - TdS + PdV
  \end{align}
  とする. 
  この1-form $\theta$をGibbs 1-formという. 
\end{defn}

この多様体$M$と1-form $\theta$を用いて熱力学的な系を定義する. 

\begin{defn}[Gibbs状態空間]
  定義1.1と定義1.2の2つ組$(M,\theta)$をGibbs状態空間という. 
\end{defn}

Gibbs状態空間の部分空間として熱力学的な平衡状態(熱平衡系)を定義する. 

\begin{defn}[滑らかな単一熱平衡系]
  $(M,\theta)$をGibbs状態空間とする. 
  滑らかな単一熱平衡系(simple smooth equilibrium thermodynamic system)とは, 2つ組$(N,\phi)$であって以下の条件を満たすものである. 
  \begin{itemize}
    % \item $N$は多様体としての2次元Euclid空間$\mbr^2$であって, $M$の$C^\infty$級部分多様体である. 
    \item $N$は$\mbr^2$の開部分多様体である. 
    \item $\phi$は滑らかなembedding $\phi: N \to M$であって
     \begin{align}
      \label{first}
      \phi^*(\theta)
      = 0
     \end{align}
    を満たす. 
    \footnote{
      $(-)^*$は写像のpullbackを表す. 
    }
    (\ref{first})を熱力学第一法則という. 
  \end{itemize}
\end{defn}

% \begin{defn}[滑らかな単一熱平衡系]
%   多様体$N$を2次元Euclid空間$\mbr^2$とする.  
%   $M$へのembeddingな写像$\phi: N \to M$が存在して
%     \begin{align}
%       \label{first}
%       \phi^*(\theta)
%       = 0
%     \end{align}
%     を満たす$M$の$C^\infty$級部分多様体$N$を滑らかな単一熱平衡系(simple smooth equilibrium thermodynamic system)という. \\
%     (\ref{first})式を熱力学第一法則(first law of thermodynamics)という. 
% \end{defn}

写像$\phi$の定義域$N$は従来の熱平衡系(equiliblium state)を表していることが分かる. 
例えば$N$のCartesian座標として$(T,V)$がとれるとする. 
\footnote{
  座標が上のようにとれる条件は後の章で補足する. 
}

この時, embedding $\phi$は
\begin{align*}
  \phi: N \to M: (T,V) \mt (U(T,V), T, S(T,V), P(T,V), V)
\end{align*}
となる. 

今は$N$のCartesian座標として$T,V$をとったが, 他に$(S,V), (S,P), (T,P)$ 
\footnote{
  勿論$(P,V)$等も考えることはできるが, 後に出てくるMaxwell関係式と関係するものだけを選んでいる. 
}
も考えることができる. 
以降はCartesian座標として$(T,V)$をとる場合のみに着目する.

この定義によって従来の熱力学の様々な関係式を満たすことを見ていく. 

\subsection{Maxwell関係式}

\begin{lem}[Maxwell関係式]
  滑らかな単一熱平衡系において
  \begin{align}
    \label{maxwell}
    \phi^*(d\theta)
    = 0
  \end{align}
  が成立する.  
  (\ref{maxwell})式をMaxwell関係式という.
\end{lem}  

\begin{prf}
  滑らかな単一熱平衡系において
  \begin{align*}
    \phi^*(\theta)
    = 0
  \end{align*}
  より
  \begin{align*}
    \phi^*(d\theta) 
    &= d \phi^*(\theta) \\
    &= 0
  \end{align*}
\end{prf}

\begin{thrm}
  Maxwell関係式(\ref{maxwell})は従来のMaxwell関係式を再現する. 
\end{thrm}

\begin{prf}
  $N$のCartesian座標として$(T,V)$がとれるとすると
  \begin{align*}
    \phi^*(d\theta) 
    &= \phi^*(-dT \wedge dS + dP \wedge dV) \\
    &= -d\phi^*(T) \wedge d\phi^*(S) + d\phi^*(P) \wedge d\phi^*(V) \\
    &= -dT \wedge \qty(\ddel{S}{T} dT + \ddel{S}{V} dV) + \qty(\ddel{P}{T} dT + \ddel{P}{V} dV) \wedge dV \\
    &= \qty(-\ddel{S}{V} + \ddel{P}{T}) dT \wedge dV \\
  \end{align*}
  (\ref{maxwell})式よりこれが$0$に等しいので
  \begin{align}
    \label{maxwell1}
    \qty(\ddel{S}{V})_T 
    = \qty(\ddel{P}{T})_V
  \end{align}
\end{prf}

$N$のCartesian座標として他の座標をとると, 他のMaxwell関係式を再現することが分かる. 

\begin{lem}
  滑らかな単一熱平衡系において, 以下の関係式が成立する. 
  \begin{align}
    \label{ene1}
    \ddel{U}{T} &= T \ddel{S}{T} \\
    \label{ene2}
    \ddel{U}{V} &= T\ddel{S}{V} - P 
  \end{align}
\end{lem}  

\begin{prf}
  (\ref{gibbs})式と(\ref{first})式より滑らかな単一熱平衡系において
  \begin{align*}
    \phi^*(d\theta) 
    &= d\phi^*(U) - Td\phi^*(S) + Pd\phi^*(V) \\
    &= \ddel{U}{T}dT + \ddel{U}{V}dV -T \qty(\ddel{S}{T}dT + \ddel{S}{V}dV) + PdV \\
    &= \qty(\ddel{U}{T} - T\ddel{S}{T})dT + \qty(\ddel{U}{V} - T\ddel{S}{V} + P)dV \\
    &= 0
  \end{align*}
  係数をそれぞれ比較して
  \begin{align*}
    \ddel{U}{T} &= T \ddel{S}{T} \\
    \ddel{U}{V} &= T\ddel{S}{V} - P 
  \end{align*}
\end{prf}

この補題よりエネルギー方程式が導かれる. 
これは従来の熱力学におけるエネルギー方程式に一致する. 

\begin{thrm}[エネルギー方程式]
  滑らかな単一熱平衡系において
  \begin{align}
    \label{ene}
    \ddel{U}{V} = T\ddel{P}{T} - P
  \end{align}
  が成立する. 
  (\ref{ene})式をエネルギー方程式という. 
\end{thrm}  

\begin{prf}
  (\ref{ene2})に(\ref{maxwell1})を代入すれば良い. 
\end{prf}

\subsection{理想気体}

次に理想気体を定義する.  

\begin{defn}[理想気体]
  $M$を熱系, $N$を滑らかな単一熱平衡系とする. 
  $N$が理想気体(ideal gas)であるとはある実数$c$と関数$f$が存在して
  \begin{align}
    \label{ideal1}
    \phi^*(PV-cT)
    = 0 \\
    \label{ideal2}
    \phi^*(U-f(T))
    = 0
  \end{align}
  を満たす時である. \\
  (\ref{ideal1})式と(\ref{ideal2})式内の
  \begin{align*}
    PV 
    = cT \\
    U
    = f(T)
  \end{align*}
  を状態方程式(equation of state)という. 
\end{defn}

% \begin{lem}[理想気体のenergyとentropy]
%   a
% \end{lem}

\newpage

\section{熱系の接触}

熱系の接触を考えるために, 2つのGibbs状態空間の直積を定義する. 

\subsection{接触}

\begin{defn}[直積熱系]
  $(M_1,\theta_1), (M_2,\theta_2)$をGibbs状態空間とする. 
  この時, 2つの熱系の直積空間$M_1 \ti M_2$を直積熱系という.

  $M_1,M_2$のCartesian座標をそれぞれ
  \begin{align*}
    (U_1,S_1,T_1,P_1,V_1) \\
    (U_2,S_2,T_2,P_2,V_2)
  \end{align*}
  とすると, 直積熱系$M_1 \ti M_2$のCartesian座標は
  \begin{align*}
    (U_1,U_2,S_1,S_2,T_1,T_2,P_1,P_2,V_1,V_2)
  \end{align*}
  と表される. 
\end{defn} 

\begin{defn}
  $M_1,M_2$上のGibbs 1-formがそれぞれ
  \begin{align*}
    \theta_1
    = dU_1 - T_1dS_1 + P_1dV_1 \\
    \theta_2
    = dU_2 - T_2dS_2 + P_2dV_2
  \end{align*}
  と表されている時, 直積熱系$M_1 \ti M_2$上の1-form $\theta'$を
  \begin{align*}
    \theta'
    &:= \theta_1 + \theta_2 \\
    &= dU_1 + dU_2 - T_1dS_1 - T_2dS_2 + P_1dV_1 + P_2dV_2
  \end{align*}
  と定義する. 
  これも単にGibbs 1-formという. 
\end{defn}
  
\begin{defn}[Gibbs直積状態空間]
  定義2.1と定義2.2の2つ組$(M_1 \ti M_2,\theta')$をGibbs直積状態空間という. 
\end{defn}

次にGibbs直積空間の部分空間を定義する. 
これにより, 熱系の接触による熱平衡系が定義される.

\begin{defn}[接触]
  $(M_1 \ti M_2,\theta')$をGibbs直積状態空間, $(N_1,\phi_1), (N_2,\phi_2)$を滑らかな単一熱平衡系とする. 
  4つ組$(N',M',\phi',\theta')$が接触(interaction pair)であるとは以下の条件を満たす時である. 
  \begin{itemize}
    \item $N'$は$N_1 \ti N_2$の開部分多様体である.
    \item $M'$は$M_1 \ti M_2$の開部分多様体である.
    \item $q_1 \in N_1, q_2 \in N_2$に対して$\phi'$は滑らかなembedding 
    \begin{align*}
      \phi': N_1 \ti N_2 \to M_1 \ti M_2: (v_1,v_2) \mt (\phi_1(v_1), \phi_2(v_2))
    \end{align*}
    であって
    \begin{align*}
      \phi'(N') \subset M'
    \end{align*}
    を満たす. 
  \end{itemize}
\end{defn}

\subsection{接触の熱平衡系の条件}

この接触の定義を用いて, 2つの熱系の接触が熱平衡系となる時の条件を考える. 

\begin{eg}[接触の熱平衡系の条件1]
  $M'$を$M_1 \ti M_2$の点$(u_1,u_2)$に対して
  \begin{align*}
    T_1(u_1)=T_2(u_2)
  \end{align*}
  を満たす$M_1 \ti M_2$の開部分多様体, 
  $N'$を$N_1 \ti N_2$の点$(v_1,v_2)$に対して
  \begin{align}
    \label{equal-t}
    T_1(\phi_1(v_1))
    = T_2(\phi_2(v_2))
  \end{align}
  を満たす$N_1 \ti N_2$の開部分多様体とする. \\
  $M'$上の関数として
  \begin{align*}
    S' &:= S_1+S_2 \\
    U' &:= U_1+U_2 \\
    T' &= T_1 = T_2 \\
    P_1':=P_1 &, P_2':=P_2 \\
    V_1':=V_1 &, V_2':=V_2
  \end{align*}
  とすると, $M'$上の1-form $\theta'$は
  \begin{align*}
    \theta'
    =dU' - T'dS' + P_1dV_1 + P_2dV_2
  \end{align*}
  となる. 
  この時, embedding $\phi':N' \to M'$は
  \begin{align*}
    \phi'^*(\theta')
    = 0
  \end{align*}
  を満たす. \\
  つまり$\phi'(N')$は滑らかな単一熱平衡系を定める. 
  これは熱系の接触による熱平衡系を表している式である.
  ここから分かるように, 平衡に関して体積$V$に束縛条件はなく, 
  (\ref{equal-t})式により温度$T$が等しいことが, 接触の熱平衡系の条件として必要であると分かる.  
\end{eg}  

\begin{eg}[接触の熱平衡系の条件2]
  $M''$を$M_1 \ti M_2$の点$(u_1,u_2)$に対して
  \begin{align*}
    T_1(u_1)=T_2(u_2)
  \end{align*}
  を満たす$M_1 \ti M_2$の開部分多様体, 
  $N''$を$N_1 \ti N_2$の点$(v_1,v_2)$に対して
  \begin{align}
    T_1(\phi_1(v_1))
    = T_2(\phi_2(v_2)) \\
    \label{equal-p}
    P_1(\phi_1(v_1))
    = P_2(\phi_2(v_2))
  \end{align}
  を満たす$N_1 \ti N_2$の開部分多様体とする. \\
  $M''$上の関数として
  \begin{align*}
    S'' &:=S_1+S_2 \\
    U'' &:=U_1+U_2 \\
    V'' &=V_1+V_2 \\
    T'' &=T_1=T_2 \\
    P'' &:=P_1=P_2 
  \end{align*}
  とすると, $M''$上の1-form $\theta''$は
  \begin{align*}
    \theta''
    =dU'' - T''dS'' + P''dV''
  \end{align*}
  となる. 
  この時, embedding $\phi'':N'' \to M''$は
  \begin{align*}
    \phi''^*(\theta'')
    = 0
  \end{align*}
  を満たす. \\
  つまり$\phi''(N'')$は滑らかな単一熱平衡系を定める. 
  これは熱系の接触による熱平衡系を表している式である.
  (\ref{equal-p})式により圧力$P$が等しいことが, 接触の熱平衡系の条件としてさらに必要であると分かる.  
\end{eg}

\newpage

\section{熱系の安定性}

この章では, Gibbs直積状態空間に関して
  \begin{align*}
    M:= M_1 ti M_2 \\
    N := \phi_1(N_1) \ti \phi_2(N_2)
  \end{align*}
  とする. 

\subsection{全エントロピーと全内部エネルギー}

$(M_1,\theta_1),(M_2,\theta_2)$をGibbs状態空間, $(M, \theta')$をGibbs直積状態空間, 
$(N',M',\phi',\theta')$を接触とする. 
$(N_1,\phi_1),(N_2,\phi_2)$をそれぞれ以下の条件
\begin{align*}
  \phi_1^*(dS_1),\phi_1^*(dV_1) \text{は線形独立である.} \\
  \phi_2^*(dS_1),\phi_2^*(dV_2) \text{は線形独立である.}
\end{align*}
を追加で満たす滑らかな単一熱平衡系とする. 

今$N_1,N_2$の座標としてそれぞれ(局所的に)$(S_1,V_1),(S_2,V_2)$がとれるとする. この時$\phi_1,\phi_2$は
\begin{align}
  \label{phi_1}
  \phi_1: (S_1,V_1) \mt (U_1(S_1,V_1), T_1(S_1,V_1), S_1, P_1(S_1,V_1), V_1) \\
  \label{phi_2}
  \phi_2: (S_2,V_2) \mt (U_2(S_2,V_2), T_2(S_2,V_2), S_2, P_2(S_2,V_2), V_2) 
\end{align}
となる. 

\begin{defn}[全エントロピーと全内部エネルギー]
  $M$上の関数として
\begin{align*}
  S &:= S_1 + S_2 \\
  U &:= U_1 + U_2
\end{align*}
と定義する. 
$S$を全エントロピー(total entropy), $U$を全内部エネルギー(total internal energy)という. 
\end{defn}

\subsection{極値定理}

この時, 全内部エネルギー$U$の極値に関する定理が導かれる. 

\begin{thrm}
  $M$上の関数$U$と$S$を開部分多様体$N$に制限したものを考える. 
  この時, 条件
  \begin{align}
    \label{sv-const}
    S = \const, V_1 = \const, V_2 = \const
  \end{align}
  を満たす開部分多様体上の$U$の極値をとる点は, $N$上の温度$T_1$と$T_2$が等しくなる点である. 
\end{thrm}

\begin{prf}
  $N$は滑らかな単一熱平衡系であるので %違う気がする
  \begin{align*}
    dU 
    &= dU_1 + dU_2 \\
    &= T_1dS_1 + T_2dS_2 - P_1dV_1 - P_2dV_2 \\
  \end{align*} 
  ここで条件\eqref{sv-const}より
  \begin{align*}
    dV_1 =dV_2 = 0 \\
    dS = dS_1 + dS_2 = 0
  \end{align*}
  であるので
  \begin{align*}
    dU 
    &= (T_1 - T_2)dS_1
  \end{align*}
  よって
  \begin{align*}
    dU = 0 
    \Leftrightarrow T_1 = T_2
  \end{align*}
\end{prf}

つまり, 2つの系が熱平衡系($T_1=T_2$)となる時, 条件\eqref{sv-const}において$U$が極値を取るという熱平衡系の安定性を表している. 

% $N_1 \ti N_2$の点$(v_1,v_2)$において
% \begin{align*}
%   T_1(\phi_1(v_1))
%     = T_2(\phi_2(v_2))
% \end{align*}
% とする

\subsection{安定性定理}

この節は本の証明が間違っているので, 矛盾が起こらないように定義して定理の説明をする. 

\begin{defn}[安定]
  $\phi_1,\phi_2$をそれぞれ\eqref{phi_1}, \eqref{phi_2}で定義された写像とする. 
  この写像が安定(stable)であるとは, 条件\eqref{sv-const}のもとで
  \begin{align}
    \label{stable}
    \ddddel{U_1}{S_1} > 0, ~\ddddel{U_2}{S_2} > 0
  \end{align}
  を満たす時である. つまり$U_1$と$U_2$がそれぞれ$S_1,S_2$について凸関数である時, 安定という. 
\end{defn}

\begin{thrm}
  $N'$を\eqref{equal-t}を満たす$N_1 \ti N_2$の開部分多様体とする. 
  条件\eqref{sv-const}において$\phi_1,\phi_2$が安定であるとき, $U$は最小値をとる.
\end{thrm}

\begin{prf}
  写像の安定性の定義より明らか. 
\end{prf}

\subsection{Legendre変換}

執筆中

\end{document}